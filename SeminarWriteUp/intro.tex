\cleardoublepage
\phantomsection
\addcontentsline{toc}{chapter}{Introduction}
\let\oldchaptermarkformat=\chaptermarkformat
\renewcommand{\chaptermarkformat}{}
\chaptermark{Introduction}
\let\chaptermarkformat=\oldchaptermarkformat
\chapter*{Introduction \label{chapter1_introduction}}

The Standard Model (SM) of particle physics~\cite{Glashow_EWK, Weinberg_EWK, Salam_EWK} has proven to describe elementary particles and their interactions successfully.
The model seperates between fermions (spin 1/2) and bosons (integer spin) particles. The bosons function as mediators of the
fundamental forces of the SM. Fermions are further subdivided into quarks and leptons. Quarks are taking part in the strong
interaction. The $\Pup$ and $\Pdown$ quark and their antiparticle partners form together with 8 different gluons (mediators
of the strong interaction) the nuclei of atoms. Leptons, such as electrons, are mainly interacting via electroweak interaction.
The bosons of the electroweak interaction are $\Pgamma$, $\PZ$ and $\PWpm$ (weak gauge bosons).

The biggest problem of the SM is that it does not contain mass terms for all massive particles, especially the $\PZ$ and $\PWpm$
bosons. Unfortunately this problem can not be solved by adding new mass terms to the SM lagrangian without losing
gauge invariance. Since the SM provides such accurate predictions for the properties of all elementary particles and their interactions,
a lot of effort was made by theorists in the 60s to save this model by adding a new mechanism, the so-called Higgs mechanism.
The mechanism itself was introduced by a couple of theorists: R. Brout, F. Englert, P. Higgs, G. S. Guralnik,
C. R. Hagen, and T. W. B. Kibble. Francois Englert and Peter Higgs were awarded with the nobel prize in 2013. The next chapter describes
the SM, the Higgs mechanism and its resulting consequences.
